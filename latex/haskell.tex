\documentclass[a4paper,9pt,twoside]{book}

\usepackage[cm]{fullpage}
\usepackage[hidelinks,
  pdfauthor={Meili Mario, Morgner Felix},
  pdftitle={Advanced Programming Paradigms --- Exercise Solution Code},
  pdfsubject={MSE},
  pdfkeywords={haskell;functional programming;advprpa}
]{hyperref}
\usepackage[outputdir=.build]{minted}
\newcommand{\hsfile}[1]{\inputminted[breaklines]{haskell}{../haskell/#1.hs}}


\title{%
Advanced Programming Paradigms\\\vspace{0.5cm}
\Large{Haskell Code}
}
\author{%
  Meili, Mario\\
  \and
  Morgner, Felix
}

\begin{document}

\maketitle{}
\tableofcontents{}

\chapter{Exercises}
\section{Exercise 1}
\hsfile{ex01/ex01}
\section{Exercise 2}
\hsfile{ex02/ex02}
\section{Exercise 3}
\hsfile{ex03/ex03}
\section{Exercise 4}
\hsfile{ex04/ex04}
\section{Exercise 5}
\hsfile{ex05/ex05}
\section{Exercise 6}
\hsfile{ex06/ex06}
\section{Exercise 7}
\hsfile{ex07/ex07}
\section{Exercise 8}
\hsfile{ex08/ex08}
\section{Exercise 9}
\hsfile{ex09/ex09}

\chapter{AdvPrPa Exam HS2010}
\section{Typing}
\hsfile{exam2010/problem_1}
\section{Polynomial}
\hsfile{exam2010/problem_2}
\section{Higher Order Functions}
\hsfile{exam2010/problem_3}
\section{Type Checking}
\hsfile{exam2010/problem_4}

\chapter{KPSp Intermediate Exam FS2011}
\section{Typen}
\hsfile{exam2011/problem_1}
\section{Pattern Matching}
\hsfile{exam2011/problem_2}
\section{Listen}
\hsfile{exam2011/problem_3}
\section{Testen}
\hsfile{exam2011/problem_4}

\chapter{KPSp Intermediate Exam FS2010}
\section{Typen}
\hsfile{exami2010/problem_1}
\section{Gültigkeit Aussagenlogischer Formeln}
\hsfile{exami2010/problem_2}
\section{Komposition}
\hsfile{exami2010/problem_3}
\section{Bäume}
\hsfile{exami2010/problem_4}

\chapter{KPSp Exam FS2009}
\section{Matrizen}
\hsfile{exam2009/problem_1}
\section{Simultane Zuweisung}
\hsfile{exam2009/problem_2}
\section{Typen}
\hsfile{exam2009/problem_3}

\chapter{KPSp Intermediate Exam FS2009}
\section{Minimum}
\hsfile{exami2009/problem_1}
\section{Typen}
\hsfile{exami2009/problem_2}
\section{Entfernen von Duplikaten}
\hsfile{exami2009/problem_3}
\section{Boolsche Ausdrücke}
\hsfile{exami2009/problem_4}

\chapter{Interpreter}
\section{Abstract Syntax Definitions}
\hsfile{interpreter/AbsSyn}
\section{Tokens and Lexer}
\hsfile{interpreter/Scanner}
\section{Generic Parsers and Combinators}
\hsfile{interpreter/ParserCombis2}
\section{Language Specific Parsers}
\hsfile{interpreter/Parser2}
\section{Interpreter}
\hsfile{interpreter/Interpreter}
\section{Main Application}
\hsfile{interpreter/Main}

\chapter{VCG}
\section{Verification Condition Generator}
\hsfile{vcg/VCG}
\section{Abstract Syntax Definitions}
\hsfile{vcg/AbsSyn}
\section{Tokens and Lexer}
\hsfile{vcg/Scanner}
\section{Generic Parsers and Combinators}
\hsfile{vcg/ParserCombis}
\section{Language Specific Parsers}
\hsfile{vcg/Parser}
\section{Interpreter}
\hsfile{vcg/Interpreter}
\section{Main Application}
\hsfile{vcg/Main}

\chapter{Slides}
\section{First Steps}
\hsfile{slides/1_first_steps}
\section{Types And Classes}
\hsfile{slides/2_types_and_classes}
\section{Defining Functions}
\hsfile{slides/3_defining_functions}
\section{List Comprehension}
\hsfile{slides/4_list_comprehension}
\section{Recursive Functions}
\hsfile{slides/5_recursive_functions}
\section{Higher-Order Functions}
\hsfile{slides/6_higher-order_functions}
\section{Declaring Types And Classes}
\hsfile{slides/7_declaring_types_and_classes}
\section{Interactive Programming}
\hsfile{slides/8_interactive_programming}
\section{Functional Parsers}
\hsfile{slides/9_functional_parsers}

\end{document}
